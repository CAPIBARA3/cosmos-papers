\documentclass[11pt]{article}

% --- Packages ---
\usepackage{amsmath,amssymb}   % math symbols
\usepackage{graphicx}           % figures
\usepackage{array}              % tables
\usepackage{booktabs}           % better tables
\usepackage{geometry}           % page layout
\usepackage{hyperref}           % clickable links
\hypersetup{
    colorlinks=true,
    linkcolor=cyan,
    filecolor=magenta,      
    urlcolor=blue,
    pdftitle={Satellite Proposal: Duo MCD},
    pdfpagemode=FullScreen,
}
\geometry{a4paper, margin=1in}

\title{Satellite Proposal: COSMOS-Duo MCD}

\author{
Joan Alcaide-Núñez\thanks{Fakultät für Physik, Ludwig-Maximilians-Universität München (LMU Munich), Geschwister-Scholl-Platz 1, 80539 Munich, Germany; \texttt{cosmos.capibara@outlook.com}} \and
Martina Solano \and
Rosa Berbejo\thanks{Facultad de Ciencias, Universidad de Salamanca (USAL), Plaza de los Caídos s/n. 37008 Salamanca, Spain} \and
Carles Fonseca Mauri\thanks{Department of Physics, University of California Los Angeles (UCLA), 475 Portola Plaza, Los Angeles, CA 90095, United States} \and
Alessandro Sabia \and
Adel Alba El Fahri \\
for the \textbf{CAPIBARA Collaboration}\footnote{Collaboration for the Analysis of Photonic and Ionic Bursts and RAdiation}}

\date{\today}

\begin{document}

\maketitle

\begin{abstract}

    This is our abstract.
    \textcolor{red}{Disclaimer: This paper was adapted from the Fly Your Satellite! proposal documentation by ESA (https://www.esa.int/Education/CubeSats_-_Fly_Your_Satellite/Fly_Your_Satellite_Proposal_Instructions).}

\end{abstract}

\textbf{Keywords:} High-energy astrophysics, CubeSat constellations, multi-messenger astronomy, student-led research, CubeSat missions, gamma-ray bursts, gravitational wave counterparts, space technology education.

\section*{Abbreviations and Acronyms}\label{sec:abbreviations_and_acronyms}

\section*{Status Overview}\label{sec:status_overview}

\begin{table}[h!]
    \centering
    \renewcommand{\arraystretch}{1.3}
    \begin{tabular}{|p{4.5cm}|c|c|c|c|p{5cm}|}
        \hline
        \textbf{Activity} & \textbf{Started (Y/N)} & \textbf{Date (mm/yy)} & \textbf{Concluded (Y/N)} & \textbf{Date (mm/yy)} & \textbf{Comments} \\
        \hline
        Concept definition \& Feasibility study & Y & 01/21 & Y & 03/21 &  \\
        \hline
        Preliminary design phase &  &  &  &  &  \\
        \hline
        Detailed design &  &  &  &  & Thermal analysis (FEA ongoing) \\
        \hline
        Structural analysis & Y & 04/22 & N &  &  \\
        \hline
        AOCS analysis &  &  &  &  &  \\
        \hline
        Other analysis &  &  &  &  & Link budget; Ground coverage \\
        \hline
        Subsystem manufacturing and testing &  &  &  &  &  \\
        \hline
        In-house developed units: prototypes manufactured and tested &  &  &  &  &  \\
        \hline
        In-house developed units: engineering/flight models manufactured and tested &  &  &  &  & OBC EM available; Payload EM tested \\
        \hline
        COTS units: procured &  &  &  &  & TTC, EPS to be procured \\
        \hline
        COTS units: functionally tested &  &  &  &  &  \\
        \hline
        Subsystem environmental testing &  &  &  &  & OBC EM random vibration \\
        \hline
        FlatSat Integration \& Testing &  &  &  &  &  \\
        \hline
        Satellite stack testing &  &  &  &  &  \\
        \hline
        Satellite fully integrated &  &  &  &  &  \\
        \hline
        System level tests performed &  &  &  &  & Day-in-the-life test ongoing; Vibration planned Q4 2024 \\
        \hline
        Flight ready &  &  &  &  &  \\
        \hline
        Launch opportunity secured &  &  &  &  & Launch window Q2 2025 \\
        \hline
        Ground station installed &  &  &  &  &  \\
        \hline
        Ground station operational &  &  &  &  &  \\
        \hline
        Other (please specify) &  &  &  &  &  \\
        \hline
    \end{tabular}
    \caption{Project Development and Verification Status}
\end{table}

%%%%%%%%%%%%%%%%%%%%%%% PART 1: MISSION DESCRIPTION %%%%%%%%%%%%%%%%%%%%%%%

\section{Mission Description}\label{sec:mission_description}

\textit{\textcolor{blue}{Formulate and give a description of the main objectives of the project, including the educational objectives and the mission objectives (technological experiment, scientific research, etc.). Summarise state-of-the-art of the subject and give existing literature. Provide a brief justification to implement this mission on a Satellite as opposed to other ground, air- or space-based platforms.}
}

\subsection{Mission Objectives}\label{sec:mission_objectives}

\textit{\textcolor{blue}{Description of the anticipated scientific or technical data products of a successful mission, what scientific/technological questions will be answered, what knowledge gaps will be filled, and/or what services will be provided.}}

\subsubsection{Mission Data Products}\label{sec:mission_data_products}

\textit{\textcolor{blue}{List the specific and concrete end-data users anticipated by the mission (e.g., if the mission deals with precision farming, saying ‘farmers’ is not enough). Explain if/how end-users are embedded in the project or if there is any collaboration scheme with organisations that will make (potential) use of the mission data products.}}

\subsubsection{End-data Users}\label{sec:end-data_users}

\textit{\textcolor{blue}{If there is any potential for business or downstream applications, these shall be highlighted in this section. Report any steps taken in that direction (contacts, potential partnerships or collaborations).}}

\subsubsection{Business or Downstream Applications}\label{sec:business_or_downstream_applications}

\textit{\textcolor{blue}{If there is any potential for business or downstream applications, these shall be highlighted in this section. Report any steps taken in that direction (contacts, potential partnerships or collaborations).}}

\subsection{Concept of Operations}\label{sec:concept_of_operations}

\textit{\textcolor{blue}{Give a description of how the mission will work in practice to meet mission objectives and the system characteristics from an operational perspective. This could include, for example, information about operational modes, on-board autonomy, mission planning, scheduling (e.g. payload activation and data recollection scheduled or activated by telecommand, etc.), attitude control and mission analysis, and the strategy for on-board data generation, storage and downlink. 
}}

\subsection{Mission Phases and Mission Timeline}\label{sec:mission_phases_and_mission_timeline}

\textit{\textcolor{blue}{Identify and briefly describe the mission timeline, including all mission phases and durations and objectives. This refers to the orbital phases and excludes the project lifecycle phases on-ground.
}}

\subsection{Mission Analysis}\label{sec:mission_analysis}

\textit{\textcolor{blue}{For the current call, launch opportunities to LEO orbits compatible with ESA zero debris approach \footnote{https://blogs.esa.int/cleanspace/2023/01/12/short-introduction-to-esas-zero-debris-approach/} are envisioned. Flexibility to several orbits is recommended as CubeSats and PocketQubes often fly on rideshare missions or as piggyback payloads and teams should be prepared to accept deployments from orbits that deviate from their optimal performance orbit.

Indicate the acceptable range of orbits to which the Satellite mission is compatible in terms of mission objectives, ground coverage, system performance, etc. and include the general mission constraints. 

Information should include altitude range, inclination, orbital lifetime requirements, other constraints such as SSO, eccentricity, RAAN/LTAN, launch window, etc. 
}}

\subsection{De-Orbiting}\label{sec:de-orbiting}

\textit{\textcolor{blue}{Describe what is the orbital lifetime and how it is compatible to the ESA zero debris approach. This means that the natural orbital decay duration shall be below 5 years (final numerical value is being consolidated at the time of opening of this Call).}}





%%%%%%%%%%%%%%%%%%%%%%% PART 2: DESIGN DEFINITION %%%%%%%%%%%%%%%%%%%%%%%

\section{Design Definition}\label{sec:design_definition}

\subsection{System Description}\label{sec:system_description}

\textit{\textcolor{blue}{Provide a block diagram of the physical architecture of the system showing the subsystems breakdown into hardware products or elements and their interconnection and interfaces. 

\begin{itemize}
    \item exploded view of the system labelling all the key components/parts.
    \item Satellite in stowed configuration showing the envelope dimensions and location of RBF pin (only for CubeSats), umbilical connector, separation spring (only for CubeSats) and deployment switches.
    \item Satellite in deployed configuration showing envelope dimensions and labels of the deployables
\end{itemize}
}}

\subsection{System Budgets}\label{sec:system_budgets}

\textit{\textcolor{blue}{Report the system budgets: for example, mass, power, pointing, thermal, link, data, delta-v, etc. indicating the margins that are applied. }}

\subsection{Payload \& Subsystems Design Definition}\label{sec:payload_and_subsystems_design_definition}

\textit{\textcolor{blue}{For every subsystem, provide a description of the physical and functional architecture, key features, operational modes (if applicable) and interfaces. Figures and schematics to be included as needed. Additional analysis and tests reports may be included in appendices.
}}

\subsubsection{Payload}\label{sec:payload}

\subsubsection{Attitude and Orbit Control (AOCS)}\label{sec:attitude_and_orbit_control_aocs}

\subsubsection{Electrical Power (EPS)}\label{sec:electrical_power_eps}

\textit{\textcolor{blue}{For EPS additionally include:
\begin{itemize}
    \item Description and graphical representation of the circuit of independent inhibits (e.g., kill switch, RBF pin, etc.), which ensure that the spacecraft is off during ground processing and launch.
    \item Information of the battery chemistry, power management circuitry and battery protections (e.g., current interrupting device, overpressure venting valves, circuit for over discharge limitation, under voltage protection, etc.).
    \item Battery qualification, safety certifications and/or flight heritage in space applications.
\end{itemize}
}}

\subsubsection{On-Board Data Handling (OBDH)}\label{sec:on-board_data_handling_obdh}

\subsubsection{On-Board Software Architecture}\label{sec:on-board_software_architecture}

\subsubsection{Telemetry, Tracking and Communications (TT\&C)}\label{sec:telemetry_tracking_and_communications_ttc}

\subsubsection{Structures}\label{sec:structures}

\subsubsection{Mechanisms}\label{sec:mechanisms}

\subsubsection{Thermal Control}\label{sec:thermal_control}

\subsubsection{Propulsion (when applicable)}\label{sec:propulsion}

\subsubsection{Grounding Scheme (EMC/EMI)}\label{sec:grounding_scheme_emc_emi}

\subsection{Ground Segment}\label{sec:ground_segment}

\textit{\textcolor{blue}{Give a description of the ground segment physical architecture, key features and locations; include figures and schematics where needed.}}


%%%%%%%%%%%%%%%%%%%%% PART 3: ASSEMBLY, INTEGRATION AND VERIFICATION %%%%%%%%%%%%%%%%%%%%%%%

\section{Assembly, Integration and Verification}\label{sec:assembly_integration_and_verification}

\subsection{Model Philosophy}\label{sec:model_philosophy}

\textit{\textcolor{blue}{Give a description of the selected model philosophy and describe which models are used (Development Model, Engineering Model, Qualification Model, Flight Model, Protoflight Model, any other model of the satellite or of some of its subassemblies) and why.
}}

\subsection{AIV Activities}\label{sec:aiv_activities}

\textit{\textcolor{blue}{Give a description of the planned main activities in the assembly, integration and verification process of the different models used, including the system level AIV activities for E(Q)M, PFM, etc. 

In case assembly, integration and test activities have been performed on subsystem level and on integrated system level, more details should be provided for each test activity including when the test has been performed, its objective, test levels and duration, and the results. List of procedures and obtained results can be added in the dedicated appendices.
}}

\subsection{Development Status Overview}\label{sec:development_status_overview}

\textit{\textcolor{blue}{Summary of the status of subsystems, payload(s) and ground segment, according to the example given in the table below. For each subsystem/ element include the development status (procurement/ manufacturing/ integration status) and, where applicable, the subsystem level tests that have been performed and specifying which tests.
For PocketQube projects, the table can be simplified to adapt to the project complexity.

\begin{table}[h!]
    \centering
    \renewcommand{\arraystretch}{1.3}
    \begin{tabular}{|p{3.5cm}|p{3cm}|p{2.5cm}|c|p{6cm}|}
        \hline
        \textbf{Subsystem / Element} & \textbf{Manufacturer (In-house / COTS)} & \textbf{Model} & \textbf{At the Uni Lab? (Y/N)} & \textbf{Status} \\
        \hline
        Payload 1 & In-house & EM & Y & 
        Manufactured and assembled; Performance test completed; Radiation test completed; Vibration test ongoing \\
        \cline{3-5}
        &  & FM & Y & 
        Same design as EM (TBC); To be manufactured \\
        \hline
        TT\&C & In-house & EM & Y & 
        HAB test campaign successfully performed \\
        \cline{3-5}
        &  & FM & N & 
        Gerber files being finalised; To be manufactured \\
        \hline
        Structure & <Company name> & FM & N & 
        CAD design finalised; Structural analysis complete; To be ordered; Qualified against random vibration and thermal vacuum / thermal cycling \\
        \hline
        Ground Segment &  &  &  &  \\
        \hline
    \end{tabular}
    \caption{Subsystems and Element Development Status}
\end{table}


}}

\subsection{Equipment Qualification Status List (for CubeSats only)}\label{sec:equipment_qualification_status_list}

\textit{\textcolor{blue}{Provide the qualification status detail of the satellite equipment/subsystems.
For PocketQube projects, the qualification status at equipment (subsystem) level is not applicable, and the spacecraft shall be qualified at once, once assembled.}}

\begin{table}[h!]
    \centering
    \renewcommand{\arraystretch}{1.3}
    \begin{tabular}{l|l|l|l|l|l|l}
        \hline
        \textbf{Item} & \textbf{Product Name \& Manufacturer} & \textbf{Model} & \textbf{Thermal Vacuum \& Cycling} & \textbf{Random Vibration} & \textbf{Shock} & \textbf{Flight Heritage} \\
        \hline
        TT\&C board & CompanyX & FM & -20/+60 degC, 6 cycles & GEVS (14.1 grms) & & \\
        \hline
         & & & & & & \\
        \hline
    \end{tabular}
    \caption{Equipment / Subsystem Qualification Status}
\end{table}

\subsection{Facilities and Ground Support Equipment}\label{sec:facilities_and_ground_support_equipment}

\textit{\textcolor{blue}{Give a short description of the support facilities (cleanroom, test facilities, etc.) available to the team and support with pictures. When applicable include the names of the organisations, departments or companies that provide in-kind support, if any.

Give a description of the test software and test tools to be utilized, and the electrical and mechanical Ground Support Equipment (EGSE + MGSE) foreseen to be used for handling on ground and tests.
}}

\subsection{Preparation for System Environmental Tests}\label{sec:preparation_for_system_environmental_tests}

\textit{\textcolor{blue}{Give a description of how the satellite will be prepared for testing in environmental conditions (shaker and thermal vacuum chamber) using the table below.
}}

\begin{table}[h!]
    \centering
    \renewcommand{\arraystretch}{1.3}
    \begin{tabular}{|p{5cm}|c|p{7cm}|}
        \hline
        \textbf{Feature} & \textbf{Y/N} & \textbf{Describe what, how and to what level} \\
        \hline
        Recharging during testing &  &  \\
        \hline
        Switching on/off during testing from GSE (also inside the thermal vacuum chamber) &  &  \\
        \hline
        Send and receive telecommands to/from the satellite (also inside the thermal vacuum chamber) &  &  \\
        \hline
        Test sensors installed & Y & 
        The satellite will have X thermocouples installed, which will be cut after the TVAC testing \\
        \hline
        Software updates possible on-ground / in-orbit after testing &  &  \\
        \hline
        Accessibility to the internal part of the satellite for executing repairs or changing components &  &  \\
        \hline
        Screws and nuts secured inside/outside the satellite &  &  \\
        \hline
        Other(s) – please specify &  &  \\
        \hline
    \end{tabular}
    \caption{Satellite Testability and Operational Features}
\end{table}


\subsection{Legal and Regulatory Status}\label{sec:legal_and_regulatory_status}

\textit{\textcolor{blue}{Provide an overview of the steps planned or undertaken considering the regulative aspects of the mission and provide evidence of the activities. 

The following reference may be useful for teams not familiar with the topic: https://swfound.org/media/188605/small_satellite_program_guide_-_chapter_5_-_legal_and_regulatory_considerations_by_chris_johnson.pdf

}}

\begin{table}
    \centering
    \renewcommand{\arraystretch}{1.3}
    \begin{tabular}{|p{6cm}|p{6cm}|}
        \hline
        \textbf{Activity (if applicable)} & \textbf{Status} \\
        \hline
        ITU frequency registration & \\
        \hline
        IARU coordination & \\
        \hline
        Licensing & \\
        \hline
        Mission Authorisation & \\
        \hline
        Insurance & \\
        \hline
        UNOOSA registration & \\
        \hline
        Export control regulations & \\
        \hline
        Space Debris Mitigation analysis & \\
        \hline
        Other(s) -please specify & \\
        \hline
    \end{tabular}
    \caption{Legal and Regulatory Status}
\end{table}

%%%%%%%%%%%%%%%%%%%%%%%%% PART 4: PROJECT ORGANISATION %%%%%%%%%%%%%%%%%%%%%%%

\section{Project Organisation}\label{sec:project_organisation}

\subsection{Team Information}\label{sec:team_information}

\textit{\textcolor{blue}{Provide the details of the supervisors of the project. A minimum of 2 supervisors shall cover the three functions. The key roles of team leader and system engineer must be undertaken by citizens of an eligible state.
}}

\textbf{Team Leader}
\begin{itemize}
    \item Name
    \item Job title or level of studies
    \item University/Tertiary Education Institution
    \item Department
\end{itemize}

\textbf{Tertiary Education Institution - Endorsing Staff}
\begin{itemize}
    \item Name
    \item Job title
    \item University/Tertiary Education Institution
    \item Department
\end{itemize}

\textbf{System Engineer}
\begin{itemize}
    \item Name
    \item Job title or level of studies
    \item University/Tertiary Education Institution
    \item Department
\end{itemize}

\textbf{List of Students}

\textit{\textcolor{blue}{Provide an overview of the key student team members currently involved in the programme. No names are required for this application, only their profile in the project.}}

\begin{table}[h!]
    \centering
    \renewcommand{\arraystretch}{1.3}
    \begin{tabular}{|p{4cm}|p{4cm}|}
        \hline
        (1) \textbf{Function} in the project (e.g., outreach, power, subsystem, soldering, machining) & (5) \textbf{# of years} in higher education so far (e.g., BSc + MSc) \\
        (2) \textbf{Involved} in the project \textbf{until} (MM YYYY) - expected date & (6) \textbf{Field of study} (e.g., aerospace engineering), \textbf{\& Specialisation} (e.g., Space Systems, Materials and Structure, etc.) \\
        (3) Preparing graduation \textbf{Thesis} within the project (Yes/No) & (7) \textbf{Nationality(s)} country code \\
        (4) Current \textbf{Level} of study (e.g., BSc/MSc/PhD) & (8) \textbf{University/Tertiary Education Institution} of enrolment \\
        \hline
    \end{tabular}
\end{table}

\begin{table}[h!]
    \centering
    \renewcommand{\arraystretch}{1.3}
    \begin{tabular}{|p{3cm}|p{2cm}|p{2cm}|p{1.8cm}|p{1.5cm}|p{4cm}|p{2cm}|p{3.5cm}|}
        \hline
        \textbf{Function} & \textbf{Involved until} & \textbf{Thesis (if any)} & \textbf{Level} & \textbf{\# of years} & \textbf{Field of study \& specialisation} & \textbf{Nationality} & \textbf{University / Institution} \\
        \hline
        AOCS Design & Q2 2025 & Yes & MSc & 5 & Aerospace Engineering; Space Systems & IT & University of Great Knowledge \\
        \hline
        System Engineer & Q4 2026 & Yes & PhD & 7 &  &  &  \\
        \hline
        PCB soldering &  &  &  &  &  &  &  \\
        \hline
        &  &  &  &  &  &  &  \\
        \hline
        &  &  &  &  &  &  &  \\
        \hline
        &  &  &  &  &  &  &  \\
        \hline
    \end{tabular}
    \caption{Team Composition and Expertise}
\end{table}

\textbf{List of early-career team members}

\textit{\textcolor{blue}{Provide an overview of the key early-career team members currently involved in the programme. No. names are required for this application, only their profile in the project.}}

\begin{table}[h!]
    \centering
    \renewcommand{\arraystretch}{1.3}
    \begin{tabular}{|p{3cm}|p{2cm}|p{2cm}|p{1.8cm}|p{1.5cm}|p{4cm}|p{2cm}|p{3.5cm}|}
        \hline
        \textbf{Function} & \textbf{Involved until} & \textbf{Thesis (if any)} & \textbf{Level} & \textbf{\# of years} & \textbf{Field of study \& specialisation} & \textbf{Nationality} & \textbf{University / Institution} \\
        \hline
        AOCS Design & Q2 2025 & Yes & MSc & 5 & Aerospace Engineering; Space Systems & IT & University of Great Knowledge \\
        \hline
        System Engineer & Q4 2026 & Yes & PhD & 7 &  &  &  \\
        \hline
        PCB soldering &  &  &  &  &  &  &  \\
        \hline
        &  &  &  &  &  &  &  \\
        \hline
        &  &  &  &  &  &  &  \\
        \hline
        &  &  &  &  &  &  &  \\
        \hline
    \end{tabular}
\caption{Team Roles and Qualifications}
\end{table}


\subsection{Organigram}\label{sec:organigram}

\textit{\textcolor{blue}{Show the responsibilities of the team members and the available manpower (e.g. weekly dedication to the project by the students).  Indicate who are students, supervisors,  professors/ experts of the university or tertiary education institution and the external parties that provide support.

In case the development of the project is shared with another party explain how the collaborations with other organisations or institutions will be practically arranged. 

Explain how the handover between different project members is organised.
}}

\subsection{Project Schedule}\label{sec:project_schedule}

\textit{\textcolor{blue}{Present the schedule of your project (e.g. in a Gantt chart), showing when the project started and including all major activities with planned milestones and tasks expected duration. The planning should include those activities for each planned model (e.g. engineering model, flight model), both at subsystem and system level. }}

\subsection{Cost Budget}\label{sec:cost_budget}

\textit{\textcolor{blue}{Provide the cost breakdown of the whole project; indicate who the funding organisations (sponsors) are, what items have secured funding and for which parts you still need to identify funding organisations. Indicate the margins between costs and funds.}}

\subsection{External Parties}\label{sec:external_parties}

\textit{\textcolor{blue}{List all partnerships and third parties involved directly or indirectly in the project, and describe their role (partners, sponsors, mentors, donators, advisors). 

Provide details of the cash or in-kind sponsorship support (hardware, software, services) and clarify if the third party expects anything in return (e.g. in-orbit data, student trainee).

Note: The list shall exclude normal COTS procurement by the university or tertiary education institution.}}

\begin{table}
    \centering
    \renewcommand{\arraystretch}{1.3}
    \begin{tabular}{|p{4cm}|p{4cm}|p{6cm}|p{3cm}|}
        \hline
        \textbf{Third Party} & \textbf{Role} & \textbf{Support} & \textbf{In return} \\
        \hline
         & & & \\
        \hline
         & & & \\
        \hline
    \end{tabular}
    \caption{External Parties Involved in the Project}
\end{table}


%%%%%%%%%%%%%%%%%%%%% PART 5: MAJOR RISKS %%%%%%%%%%%%%%%%%%%%%%%


\section{Major Risks}\label{sec:major_risks}

\textit{\textcolor{blue}{Give a list of the major risks to the project both in technical and managerial terms, and an assessment of the follow-up actions/mitigations plans. Risks may be linked to e.g., reliability, radiation, single point failures, lack of funding, project schedule, procurement delays, manpower, student turnovers, lack of experience in certain areas. 
}}


%%%%%%%%%%%%%%%%%%%%% PART 6. ACADEMIC RETURN %%%%%%%%%%%%%%%%%%%%%%%

\section{Academic Return}\label{sec:academic_return}

\subsection{Institution Experience and Background}\label{sec:institution_experience_and_background}

\textit{\textcolor{blue}{Give a brief description of the university or tertiary education institution’s experience in building educational satellites, if any, or in running other sorts of educational space-related hands-on projects. }}

\subsection{Educational and Academic Return of the Project}\label{sec:educational_and_academic_return_of_the_project}

\textit{\textcolor{blue}{Give a description of how the satellite is used for educational purposes in the university or tertiary education institution and how the project will be included in the syllabus of students (e.g. part of master or PhD thesis, a research programme, or any form of project supported by the applicant’s university, etc.). Is there any formal recognition of the student work on the project (e.g. ECTS)? Quantify whenever possible, the outputs of the project. If more students are expected to work on the project than included in the proposal please indicate the total number and over what period. }}

\subsection{Proposal Motivation}\label{sec:proposal_motivation}

\textit{\textcolor{blue}{Give a description of the main motivations for applying to the Fly Your Satellite! programme and what kind of support the team is seeking for in the programme. }}

%%%%%%%%%%%%%%%%%%%%%%%%%%%% REFERENCES %%%%%%%%%%%%%%%%%%%%%%%%%%

\bibliographystyle{unsrt}
\bibliography{references}


%%%%%%%%%%%%%%%%%%%%%%%%%%%% APPENDICES %%%%%%%%%%%%%%%%%%%%%%%%%%%%

\appendix

\section*{Appendices}\label{sec:appendices}

\subsection*{Appendix A: Technical Requirements Specification}\label{sec:appendix_a_technical_requirements_specification}

\textit{\textcolor{blue}{List all the technical requirements including:
\begin{itemize}
    \item \textbf{Requirement identification}: all requirements should be identified with a unique identifier.
    \item \textbf{Requirement text}: definition of the requirement. Requirements should be SMART: Specific, Measurable, Achievable, Relevant and Traceable.
\end{itemize}

Technical Specification may be attached in any format (also in case it includes more information beyond the Requirement text, such as verification strategy, parent/child relations, etc.).
}}

\begin{table}[h!]
    \centering
    \renewcommand{\arraystretch}{1.3}
    \begin{tabular}{|p{3cm}|p{10cm}|}
        \hline
        \textbf{Requirement ID} & \textbf{Requirement Text} \\
        \hline
         & \\
        \hline
         & \\
        \hline
    \end{tabular}
    \caption{Technical Requirements Specification}
\end{table}

\subsection*{Appendix B: Technical Requirements — Compliance Matrix}\label{sec:appendix_b_technical_requirements_compliance_matrix}

\textit{\textcolor{blue}{Attach the filled in 'Fly Your Satellite! Design Specification — Compliance Matrix' (CUbeSats or PocketQubes) in this appendix.}}

We don't have access to that file, so we leave this section blank.

\subsection*{Appendix C: Pictures}\label{sec:pictures}

\textit{\textcolor{blue}{Pictures of satellite subsystems/components, antennas and ground station. Include in the caption which model (e.g., FM, EM, prototype) is shown in the picture.}}

\begin{figure}[h!]
    \centering
    \includegraphics[width=0.5\textwidth]{example-image}
    \caption{Example caption describing the image shown.}
    \label{fig:example}
\end{figure}

\subsection*{Other Appendices}\label{sec:other_appendices}

\textit{\textcolor{blue}{
    \begin{itemize}
        \item Analysis reports (if available)
        \item Test reports (if available)
        \item Assembly and integration reports (if available)
        \item Space Debris Mitigation Analysis  (if available)
    \end{itemize}
}}

\end{document}
